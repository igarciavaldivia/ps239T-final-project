\documentclass{beamer}	%So you want to make a presentation in LaTeX using beamer
\mode<presentation> {
	\usetheme[sectionpage=none]{metropolis} %Eliminates progress slide before each section
	%\usetheme{metropolis}	% Use metropolis theme basic
}

%Load Packages
\usepackage{hyperref} %lets you embed hyperlinks into your document
\usepackage{graphicx} %lets you add pictures
\graphicspath{{/Users/isabelgarcia/Desktop/ps239T-final-project/Presentation/}} %set location of pictures
\usepackage{subcaption} %adds captions and multiple figures
\usepackage{caption}
\captionsetup{font=footnotesize} %change the font of captions
\usepackage[textfont={scriptsize}]{caption} %used to make captions smaller throughout the presentation unless you override
\usepackage{graphicx} %used to make table smaller


%We can add our title info here:
\title{Aging Immigrant Population by Legal Status}
\date{\today}	% set automatically to today's date
\author{Isabel Garc\'ia Valdivia}
\institute{University of California, Berkeley\\
Department of Sociology}

\begin{document}
  \maketitle
  %Making a table of contents is still super easy
  \tableofcontents
  \clearpage
  
\section{Research Question}
  \begin{frame}{Research Question}
What are the socio-demographic characteristics of older (aging) adults in the U.S.?
\begin{itemize}
	\item How does it differ by legal status and gender?
	\item Compared to Mexicans and Mexican-origin immigrants?
\end{itemize}
  \end{frame}

\section{Data and Analysis}
 	\begin{frame}{Data}
		\begin{itemize}
			\item Dataset
				\begin{itemize}
					\item Current Population Survey\textsc{\char13}s (CPS) Annual Social and Economic Supplement (ASEC), 2016 and 2017
				\end{itemize}
			\item Tools
				\begin{itemize}
					\item Processing and Visualizations in R
					\item Machine Learning in R
					\item Presentation in LaTeX
				\end{itemize}
		\end{itemize}
	\end{frame}
	\begin{frame}{Analysis}
		\begin{itemize}
			\item Descriptive Analysis
			\item Imputation of Legal Status
				\begin{itemize}
					\item Residual Method (Bachmeier, Van Hook, and Bean 2014; Passel 2007; Passel and Cohn 2009; Warren 2014) using CPS 2017
					\item Machine Learning: Classification
					\begin{itemize}
						\item CPS 2017 as training data
						\item CPS 2016 as test data
					\end{itemize}
				\end{itemize}
		\end{itemize}
	\end{frame}

\section{Results}

\begin{frame}{Results}
\begin{table}[ht]
	\centering
Older Adults by Legal Status
\scalebox{0.8}{
\begin{tabular}{llrrrr}
	\hline
	& Age(s) & legal immigrants & undocumented & us-born \\ 
	\hline
	& 50 to 54 years & 22.1\% & 40.1\% & 18.2\% &  \\ 
	& 55 to 59 years & 19.2\% & 27.3\% & 19.4\% &  \\ 
	& 60 to 64 years & 16.2\% & 19.7\% & 17.5\% &  \\ 
	& 65 and over & 42.5\% & 12.9\% & 44.8\% &  \\ 
	& Total  & 100.0\% & 100.0\% & 100.0\% &  \\ 
	& \\
	& \% Within Older Adults & 13.0\% & 1.6\% & 85.4\% &  \\ 
	\hline
	\end{tabular}}
	\caption{Author's Tabulations. Source: IPUMS Current Population Survey, 2017}
\end{table}
\end{frame}

\begin{frame}{Results}
	\begin{figure}[h]
		\centering
		\includegraphics[width=4in]{{g1_boxplot.png}}
		\caption{Distribution of Older Adults (50 years and older) in the U.S. by Legal Status (Source: IPUMS Current Population Survey, 2017)}
	\end{figure}
\end{frame}

\begin{frame}{Results}
\begin{figure}[h]
	\centering
	\includegraphics[width=4in]{{g2_undocimmigrants.png}}
	\caption{Undocumented Older Adults (50 years old and older) in the U.S. by Age and Gender, 1.8 million (Source: IPUMS Current Population Survey, 2017)}
\end{figure}
\end{frame}

\begin{frame}{Results}
\begin{figure}[h] % h = Place the float here
	\centering
	\captionsetup[subfigure]{font=scriptsize,labelfont=scriptsize} %changes the subcaption into smaller font in these tables only
	\begin{subfigure}[b]{0.4\linewidth}
		\includegraphics[width=\linewidth]{{g1_legalimmigrants.png}}
		\caption{Legal Immigrants (3.8 million).}
	\end{subfigure}
	\begin{subfigure}[b]{0.4\linewidth}
		\includegraphics[width=\linewidth]{{g2_undocimmigrants.png}}
		\caption{Undocumented (1.8 million).}
	\end{subfigure}
	\begin{subfigure}[b]{0.4\linewidth}
		\includegraphics[width=\linewidth]{{g3_usborn.png}}
		\caption{U.S. Born (95.6 million).}
	\end{subfigure}
	\caption{Older Adults (50 years old and older) in the U.S. by Age, Gender, and Legal Status (Source: IPUMS Current Population Survey, 2017)}
	\label{fig:Older Adults}
\end{figure}
\end{frame}

\begin{frame}{Results}
\begin{figure}[h] % h = Place the float here
	\centering
	\captionsetup[subfigure]{font=scriptsize,labelfont=scriptsize} %changes the subcaption into smaller font in these tables only
	\begin{subfigure}[b]{0.4\linewidth}
		\includegraphics[width=\linewidth]{{g5_legalmex.png}}
		\caption{Legal Immigrants (3 million).}
	\end{subfigure}
	\begin{subfigure}[b]{0.4\linewidth}
		\includegraphics[width=\linewidth]{{g6_undocmex.png}}
		\caption{Undocumented (700,000).}
	\end{subfigure}
	\begin{subfigure}[b]{0.4\linewidth}
		\includegraphics[width=\linewidth]{{g4_usbornmex.png}}
		\caption{U.S. Born (2.9 million).}
	\end{subfigure}
	\caption{Mexican Older Adults (50 years old and older) in the U.S. by Age, Gender, and Legal Status (Source: IPUMS Current Population Survey, 2017)}
	\label{fig:Mexican Older Adults}
\end{figure}
\end{frame}

\begin{frame}{Results}
\begin{columns}[T] % align columns
\begin{column}{.48\textwidth}
\textbf{Machine Learning in R}
Influential Variables
	\begin{itemize}
		\item Citizenship (*Variable of most importance)
		\item Year of migration
		\item Recipient of Medicare 
		\item Recipient of Medicaid
		\item Citizenship of spouse
		\item Birthplace
		\item Recipient of Food Stamp
		\item Occupation
		\item Class of worker
		\item Veteran Status
	\end{itemize}
\end{column}%
\begin{column}{.48\textwidth}
	\textbf{Decision Tree Diagram}
	\begin{figure}[h] % h = Place the float here
		\centering
		\includegraphics[width=3in]{{binary_model.png}}
		\caption{Decision Tree. (Source: IPUMS Current Population Survey, 2017)}
	\end{figure}
\end{column}%
\end{columns}
\end{frame}

\begin{frame}{Results}
\textbf{Influential Variables: Output}
\centering
\includegraphics[width=4.5in]{{decision_tree_output.png}}
\end{frame}

\section{Challenges}
  \begin{frame}{Challenges}
	\begin{itemize}
		\item Theoretical Assumptions of Imputation of Legal Status
		\item Machine Learning Method
	\end{itemize}
\end{frame}

\section{Next Steps}
\begin{frame}{Next Steps}
\begin{itemize}
	\item Re-weighting data (over- and undercount)
	\item Continue to explore best method for machine learning 
	\item Apply to more years, CPS AESC (2016 and prior years)
	\item Apply these methods to the American Community Survey (ACS; 1\% sample or 3 million households) because it has more observations. I am focusing on a very specific group.
\end{itemize}
\end{frame}

\section{Appendix}
\begin{frame} [shrink=20]{Appendix} %shrink - makes font smaller
Residual Method Variables
\begin{itemize}
	\item That person arrived before 1980;
	\item That person is a citizen;
	\item That person receives Social Security benefits, SSI, Medicaid, Medicare, or Military Insurance;
	\item That person is a veteran, or is currently in the Armed Forces;
	\item That person works in the government sector;
	\item That person resides in public housing or receives rental subsidies, or that person is a spouse of someone who resides in public housing or receives rental subsidies;
	\item That person was born in Cuba (as practically all Cuban immigrants were granted refugee status before 2017);
	\item That person's occupation requires some form of licensing (such as physicians, registered nurses, air traffic controllers, and lawyers);
	\item That person's spouse is a legal immigrant or citizen.
\end{itemize}
\end{frame}

\end{document}